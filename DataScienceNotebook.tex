% Options for packages loaded elsewhere
\PassOptionsToPackage{unicode}{hyperref}
\PassOptionsToPackage{hyphens}{url}
%
\documentclass[
]{article}
\usepackage{lmodern}
\usepackage{amssymb,amsmath}
\usepackage{ifxetex,ifluatex}
\ifnum 0\ifxetex 1\fi\ifluatex 1\fi=0 % if pdftex
  \usepackage[T1]{fontenc}
  \usepackage[utf8]{inputenc}
  \usepackage{textcomp} % provide euro and other symbols
\else % if luatex or xetex
  \usepackage{unicode-math}
  \defaultfontfeatures{Scale=MatchLowercase}
  \defaultfontfeatures[\rmfamily]{Ligatures=TeX,Scale=1}
\fi
% Use upquote if available, for straight quotes in verbatim environments
\IfFileExists{upquote.sty}{\usepackage{upquote}}{}
\IfFileExists{microtype.sty}{% use microtype if available
  \usepackage[]{microtype}
  \UseMicrotypeSet[protrusion]{basicmath} % disable protrusion for tt fonts
}{}
\makeatletter
\@ifundefined{KOMAClassName}{% if non-KOMA class
  \IfFileExists{parskip.sty}{%
    \usepackage{parskip}
  }{% else
    \setlength{\parindent}{0pt}
    \setlength{\parskip}{6pt plus 2pt minus 1pt}}
}{% if KOMA class
  \KOMAoptions{parskip=half}}
\makeatother
\usepackage{xcolor}
\IfFileExists{xurl.sty}{\usepackage{xurl}}{} % add URL line breaks if available
\IfFileExists{bookmark.sty}{\usepackage{bookmark}}{\usepackage{hyperref}}
\hypersetup{
  pdftitle={Data Science Notebook},
  hidelinks,
  pdfcreator={LaTeX via pandoc}}
\urlstyle{same} % disable monospaced font for URLs
\usepackage[margin=1in]{geometry}
\usepackage{color}
\usepackage{fancyvrb}
\newcommand{\VerbBar}{|}
\newcommand{\VERB}{\Verb[commandchars=\\\{\}]}
\DefineVerbatimEnvironment{Highlighting}{Verbatim}{commandchars=\\\{\}}
% Add ',fontsize=\small' for more characters per line
\usepackage{framed}
\definecolor{shadecolor}{RGB}{248,248,248}
\newenvironment{Shaded}{\begin{snugshade}}{\end{snugshade}}
\newcommand{\AlertTok}[1]{\textcolor[rgb]{0.94,0.16,0.16}{#1}}
\newcommand{\AnnotationTok}[1]{\textcolor[rgb]{0.56,0.35,0.01}{\textbf{\textit{#1}}}}
\newcommand{\AttributeTok}[1]{\textcolor[rgb]{0.77,0.63,0.00}{#1}}
\newcommand{\BaseNTok}[1]{\textcolor[rgb]{0.00,0.00,0.81}{#1}}
\newcommand{\BuiltInTok}[1]{#1}
\newcommand{\CharTok}[1]{\textcolor[rgb]{0.31,0.60,0.02}{#1}}
\newcommand{\CommentTok}[1]{\textcolor[rgb]{0.56,0.35,0.01}{\textit{#1}}}
\newcommand{\CommentVarTok}[1]{\textcolor[rgb]{0.56,0.35,0.01}{\textbf{\textit{#1}}}}
\newcommand{\ConstantTok}[1]{\textcolor[rgb]{0.00,0.00,0.00}{#1}}
\newcommand{\ControlFlowTok}[1]{\textcolor[rgb]{0.13,0.29,0.53}{\textbf{#1}}}
\newcommand{\DataTypeTok}[1]{\textcolor[rgb]{0.13,0.29,0.53}{#1}}
\newcommand{\DecValTok}[1]{\textcolor[rgb]{0.00,0.00,0.81}{#1}}
\newcommand{\DocumentationTok}[1]{\textcolor[rgb]{0.56,0.35,0.01}{\textbf{\textit{#1}}}}
\newcommand{\ErrorTok}[1]{\textcolor[rgb]{0.64,0.00,0.00}{\textbf{#1}}}
\newcommand{\ExtensionTok}[1]{#1}
\newcommand{\FloatTok}[1]{\textcolor[rgb]{0.00,0.00,0.81}{#1}}
\newcommand{\FunctionTok}[1]{\textcolor[rgb]{0.00,0.00,0.00}{#1}}
\newcommand{\ImportTok}[1]{#1}
\newcommand{\InformationTok}[1]{\textcolor[rgb]{0.56,0.35,0.01}{\textbf{\textit{#1}}}}
\newcommand{\KeywordTok}[1]{\textcolor[rgb]{0.13,0.29,0.53}{\textbf{#1}}}
\newcommand{\NormalTok}[1]{#1}
\newcommand{\OperatorTok}[1]{\textcolor[rgb]{0.81,0.36,0.00}{\textbf{#1}}}
\newcommand{\OtherTok}[1]{\textcolor[rgb]{0.56,0.35,0.01}{#1}}
\newcommand{\PreprocessorTok}[1]{\textcolor[rgb]{0.56,0.35,0.01}{\textit{#1}}}
\newcommand{\RegionMarkerTok}[1]{#1}
\newcommand{\SpecialCharTok}[1]{\textcolor[rgb]{0.00,0.00,0.00}{#1}}
\newcommand{\SpecialStringTok}[1]{\textcolor[rgb]{0.31,0.60,0.02}{#1}}
\newcommand{\StringTok}[1]{\textcolor[rgb]{0.31,0.60,0.02}{#1}}
\newcommand{\VariableTok}[1]{\textcolor[rgb]{0.00,0.00,0.00}{#1}}
\newcommand{\VerbatimStringTok}[1]{\textcolor[rgb]{0.31,0.60,0.02}{#1}}
\newcommand{\WarningTok}[1]{\textcolor[rgb]{0.56,0.35,0.01}{\textbf{\textit{#1}}}}
\usepackage{graphicx}
\makeatletter
\def\maxwidth{\ifdim\Gin@nat@width>\linewidth\linewidth\else\Gin@nat@width\fi}
\def\maxheight{\ifdim\Gin@nat@height>\textheight\textheight\else\Gin@nat@height\fi}
\makeatother
% Scale images if necessary, so that they will not overflow the page
% margins by default, and it is still possible to overwrite the defaults
% using explicit options in \includegraphics[width, height, ...]{}
\setkeys{Gin}{width=\maxwidth,height=\maxheight,keepaspectratio}
% Set default figure placement to htbp
\makeatletter
\def\fps@figure{htbp}
\makeatother
\setlength{\emergencystretch}{3em} % prevent overfull lines
\providecommand{\tightlist}{%
  \setlength{\itemsep}{0pt}\setlength{\parskip}{0pt}}
\setcounter{secnumdepth}{-\maxdimen} % remove section numbering

\title{Data Science Notebook}
\author{}
\date{\vspace{-2.5em}}

\begin{document}
\maketitle

\hypertarget{r-programming}{%
\section{R Programming}\label{r-programming}}

\hypertarget{loop-functions}{%
\subsection{Loop Functions}\label{loop-functions}}

\begin{itemize}
\item
  lapply: Loop over a \textbf{list} and evaluate a function on each
  element
\item
  sapply: Same as lapply but try simplify the result
\item
  apply: Apply a function over the margins of an array
\item
  tapply(table apply): Apply a function over subsets of a vector
\item
  mapply: Multivariate version of apply
\end{itemize}

\hypertarget{lapply}{%
\subsubsection{lapply}\label{lapply}}

lapply \textbf{always} return a list, regardless of the class of the
input

\begin{Shaded}
\begin{Highlighting}[]
\NormalTok{x \textless{}{-}}\StringTok{ }\KeywordTok{list}\NormalTok{(}\DataTypeTok{a =} \DecValTok{1}\OperatorTok{:}\DecValTok{4}\NormalTok{, }\DataTypeTok{b =} \KeywordTok{rnorm}\NormalTok{(}\DecValTok{10}\NormalTok{), }\DataTypeTok{c =} \KeywordTok{rnorm}\NormalTok{(}\DecValTok{20}\NormalTok{, }\DecValTok{1}\NormalTok{), }\DataTypeTok{d =} \KeywordTok{rnorm}\NormalTok{(}\DecValTok{100}\NormalTok{, }\DecValTok{5}\NormalTok{))}
\KeywordTok{lapply}\NormalTok{(x, mean)}
\end{Highlighting}
\end{Shaded}

\begin{verbatim}
## $a
## [1] 2.5
## 
## $b
## [1] -0.1278185
## 
## $c
## [1] 1.028772
## 
## $d
## [1] 5.00698
\end{verbatim}

\begin{Shaded}
\begin{Highlighting}[]
\NormalTok{x \textless{}{-}}\StringTok{ }\DecValTok{1}\OperatorTok{:}\DecValTok{4}
\KeywordTok{lapply}\NormalTok{(x, runif) }\CommentTok{\# runif:生成给定个数的随机数}
\end{Highlighting}
\end{Shaded}

\begin{verbatim}
## [[1]]
## [1] 0.2338
## 
## [[2]]
## [1] 0.3822538 0.6187757
## 
## [[3]]
## [1] 0.9436122 0.7727679 0.4335463
## 
## [[4]]
## [1] 0.35048781 0.47884682 0.17664804 0.02274831
\end{verbatim}

\begin{Shaded}
\begin{Highlighting}[]
\NormalTok{x \textless{}{-}}\StringTok{ }\DecValTok{1}\OperatorTok{:}\DecValTok{4}
\KeywordTok{lapply}\NormalTok{(x, runif, }\DataTypeTok{min =} \DecValTok{0}\NormalTok{, }\DataTypeTok{max =} \DecValTok{10}\NormalTok{) }\CommentTok{\# 设定最小值和最大值}
\end{Highlighting}
\end{Shaded}

\begin{verbatim}
## [[1]]
## [1] 4.000257
## 
## [[2]]
## [1] 8.547951 6.009444
## 
## [[3]]
## [1] 1.103804 1.404902 1.329846
## 
## [[4]]
## [1] 0.3525163 7.9086601 2.4990813 1.6199491
\end{verbatim}

lapply and friends make heavy use of \emph{anonymous} functions.

\begin{Shaded}
\begin{Highlighting}[]
\NormalTok{x \textless{}{-}}\StringTok{ }\KeywordTok{list}\NormalTok{(}\DataTypeTok{a =} \KeywordTok{matrix}\NormalTok{(}\DataTypeTok{data =} \DecValTok{1}\OperatorTok{:}\DecValTok{4}\NormalTok{, }\DataTypeTok{nrow =} \DecValTok{2}\NormalTok{, }\DataTypeTok{ncol =} \DecValTok{2}\NormalTok{), }\DataTypeTok{b =} \KeywordTok{matrix}\NormalTok{(}\DecValTok{1}\OperatorTok{:}\DecValTok{6}\NormalTok{, }\DecValTok{3}\NormalTok{, }\DecValTok{2}\NormalTok{))}
\KeywordTok{lapply}\NormalTok{(x, }\ControlFlowTok{function}\NormalTok{(elt) elt[,}\DecValTok{1}\NormalTok{])}
\end{Highlighting}
\end{Shaded}

\begin{verbatim}
## $a
## [1] 1 2
## 
## $b
## [1] 1 2 3
\end{verbatim}

\hypertarget{sapply}{%
\subsubsection{sapply}\label{sapply}}

sapply will to simplify the result of lapply if possible

\begin{itemize}
\item
  If the result is a list where every element is length 1, then a vector
  is returned.
\item
  If the result is a list where every element is a vecotr of the same
  length (\textgreater1), a matrix is returned.
\item
  If it can't figure things out,a list is returned.
\end{itemize}

\begin{Shaded}
\begin{Highlighting}[]
\NormalTok{x \textless{}{-}}\StringTok{ }\KeywordTok{list}\NormalTok{(}\DataTypeTok{a =} \DecValTok{1}\OperatorTok{:}\DecValTok{4}\NormalTok{, }\DataTypeTok{b =} \KeywordTok{rnorm}\NormalTok{(}\DecValTok{10}\NormalTok{), }\DataTypeTok{c =} \KeywordTok{rnorm}\NormalTok{(}\DecValTok{20}\NormalTok{,}\DecValTok{1}\NormalTok{), }\DataTypeTok{d =} \KeywordTok{rnorm}\NormalTok{(}\DecValTok{100}\NormalTok{,}\DecValTok{5}\NormalTok{))}
\KeywordTok{sapply}\NormalTok{(x, mean)}
\end{Highlighting}
\end{Shaded}

\begin{verbatim}
##          a          b          c          d 
## 2.50000000 0.02184388 1.05973645 5.06423436
\end{verbatim}

\hypertarget{apply}{%
\subsubsection{apply}\label{apply}}

apply is userd to a evaluate function (often an anoymous one) over the
margins of an array

\begin{itemize}
\item
  It is most often used to apply a function to the rows or columns of a
  matrix
\item
  It can be used with general arrays, e.g.~taking the average of an
  array of matrices
\item
  It is not really faster than writing a loop, but it works in one line!
\end{itemize}

\begin{Shaded}
\begin{Highlighting}[]
\KeywordTok{str}\NormalTok{(apply)}
\end{Highlighting}
\end{Shaded}

\begin{verbatim}
## function (X, MARGIN, FUN, ...)
\end{verbatim}

\begin{itemize}
\item
  X is an array
\item
  \textbf{MARGIN} is an integer vector indicating which margins should
  be ``retained''
\item
  \textbf{FUN} is a function to be applied
\item
  \ldots{} is for other auguments to be passed to FUN
\end{itemize}

\begin{Shaded}
\begin{Highlighting}[]
\NormalTok{x \textless{}{-}}\StringTok{ }\KeywordTok{matrix}\NormalTok{(}\KeywordTok{rnorm}\NormalTok{(}\DecValTok{200}\NormalTok{), }\DecValTok{20}\NormalTok{, }\DecValTok{10}\NormalTok{)}

\CommentTok{\# 2 means the second dimension}
\KeywordTok{apply}\NormalTok{(x, }\DecValTok{2}\NormalTok{, mean) }\CommentTok{\# calculate the mean of each columns of the matrix}
\end{Highlighting}
\end{Shaded}

\begin{verbatim}
##  [1]  0.112345792  0.258649726 -0.049758164 -0.227022900  0.026803401
##  [6] -0.001736036  0.015822215  0.106649477  0.088589313 -0.086457545
\end{verbatim}

\begin{Shaded}
\begin{Highlighting}[]
\KeywordTok{apply}\NormalTok{(x, }\DecValTok{1}\NormalTok{, sum) }\CommentTok{\# the sum of each rows of the matrix}
\end{Highlighting}
\end{Shaded}

\begin{verbatim}
##  [1] -1.31816180  2.98064704  3.05921487 -2.03404282  1.12849560  0.33589588
##  [7] -2.91688654 -3.16319513  3.73967981  0.46405728  0.26845214  1.91650232
## [13] -3.77992861 -2.42457961 -1.22057013  3.50428218 -0.08944995  3.07767107
## [19] -1.70153601  3.05115802
\end{verbatim}

\hypertarget{colrow-sums-and-means}{%
\paragraph{col/row sums and means}\label{colrow-sums-and-means}}

For sums and means of matrix dimensions, we have some shortcuts.

\begin{itemize}
\item
  rowSums = apply(x, 1, sum)
\item
  rowMeans = apply(x, 1, mean)
\item
  colSums = apply(x, 2, sum)
\item
  colMeans = apply(x, 2, mean)
\end{itemize}

The shortcut function are \emph{much} faster, but you won't notice
unless you're using a large matrix.

\hypertarget{other-ways-to-apply}{%
\paragraph{Other ways to Apply}\label{other-ways-to-apply}}

Quantiles of the rows of a matrix

\begin{Shaded}
\begin{Highlighting}[]
\NormalTok{x \textless{}{-}}\StringTok{ }\KeywordTok{matrix}\NormalTok{(}\KeywordTok{rnorm}\NormalTok{(}\DecValTok{200}\NormalTok{), }\DecValTok{20}\NormalTok{, }\DecValTok{10}\NormalTok{)}
\KeywordTok{apply}\NormalTok{(x, }\DecValTok{1}\NormalTok{, quantile, }\DataTypeTok{probs =} \KeywordTok{c}\NormalTok{(}\FloatTok{0.25}\NormalTok{, }\FloatTok{0.75}\NormalTok{))}
\end{Highlighting}
\end{Shaded}

\begin{verbatim}
##          [,1]       [,2]       [,3]       [,4]       [,5]       [,6]       [,7]
## 25% -1.269043 -1.0292604 -0.1698844 -0.2612843 -1.1260270 -0.3123889 -0.7465558
## 75%  1.274919  0.6802779  1.0588464  0.4415585  0.9484556  0.5304493  0.5017373
##           [,8]       [,9]       [,10]      [,11]       [,12]      [,13]
## 25% -0.8509583 -0.4208479 -1.65354904 -0.3415534 -0.80634711 -0.4706884
## 75%  0.2823195  0.6478099 -0.02789911  0.4598912  0.08947282  0.8809988
##          [,14]      [,15]      [,16]      [,17]      [,18]        [,19]
## 25% -0.5284515 -0.8767446 -0.6221841 -0.8895972 -0.4368462 -1.525009953
## 75%  0.5616398  0.2314100  1.3038625  0.4518838  0.8541767  0.008835564
##          [,20]
## 25% -0.7965885
## 75%  0.4413160
\end{verbatim}

Average matrix in an array

\begin{Shaded}
\begin{Highlighting}[]
\NormalTok{a \textless{}{-}}\StringTok{ }\KeywordTok{array}\NormalTok{(}\KeywordTok{rnorm}\NormalTok{(}\DecValTok{2} \OperatorTok{*}\StringTok{ }\DecValTok{2} \OperatorTok{*}\StringTok{ }\DecValTok{10}\NormalTok{), }\KeywordTok{c}\NormalTok{(}\DecValTok{2}\NormalTok{, }\DecValTok{2}\NormalTok{, }\DecValTok{10}\NormalTok{))}
\KeywordTok{apply}\NormalTok{(a, }\KeywordTok{c}\NormalTok{(}\DecValTok{1}\NormalTok{,}\DecValTok{2}\NormalTok{), mean)}
\end{Highlighting}
\end{Shaded}

\begin{verbatim}
##            [,1]        [,2]
## [1,]  0.6164147  0.04601830
## [2,] -0.1932988 -0.04778363
\end{verbatim}

\begin{Shaded}
\begin{Highlighting}[]
\KeywordTok{rowMeans}\NormalTok{(a, }\DataTypeTok{dim =} \DecValTok{2}\NormalTok{)}
\end{Highlighting}
\end{Shaded}

\begin{verbatim}
##            [,1]        [,2]
## [1,]  0.6164147  0.04601830
## [2,] -0.1932988 -0.04778363
\end{verbatim}

\hypertarget{mapply}{%
\subsubsection{mapply}\label{mapply}}

\textbf{mapply} is a multivariate apply of sorts which applies a
function in parallel over a set of arguments.

\begin{Shaded}
\begin{Highlighting}[]
\KeywordTok{str}\NormalTok{(mapply)}
\end{Highlighting}
\end{Shaded}

\begin{verbatim}
## function (FUN, ..., MoreArgs = NULL, SIMPLIFY = TRUE, USE.NAMES = TRUE)
\end{verbatim}

\begin{itemize}
\tightlist
\item
  \textbf{FUN} is a function to apply
\item
  \ldots{} contains arguments to apply over
\item
  \textbf{MoreArgs} is a list of other arguments to \textbf{FUN}
\item
  \textbf{SIMPLIFY} indicates wheather the result should be simplified
\end{itemize}

The following is tedious to type

\begin{Shaded}
\begin{Highlighting}[]
\KeywordTok{list}\NormalTok{(}\KeywordTok{rep}\NormalTok{(}\DecValTok{1}\NormalTok{, }\DecValTok{4}\NormalTok{), }\KeywordTok{rep}\NormalTok{(}\DecValTok{2}\NormalTok{, }\DecValTok{3}\NormalTok{), }\KeywordTok{rep}\NormalTok{(}\DecValTok{3}\NormalTok{, }\DecValTok{2}\NormalTok{), }\KeywordTok{rep}\NormalTok{(}\DecValTok{4}\NormalTok{, }\DecValTok{1}\NormalTok{))}
\end{Highlighting}
\end{Shaded}

\begin{verbatim}
## [[1]]
## [1] 1 1 1 1
## 
## [[2]]
## [1] 2 2 2
## 
## [[3]]
## [1] 3 3
## 
## [[4]]
## [1] 4
\end{verbatim}

Instead we can do

\begin{Shaded}
\begin{Highlighting}[]
\KeywordTok{mapply}\NormalTok{(rep, }\DecValTok{1}\OperatorTok{:}\DecValTok{4}\NormalTok{, }\DecValTok{4}\OperatorTok{:}\DecValTok{1}\NormalTok{)}
\end{Highlighting}
\end{Shaded}

\begin{verbatim}
## [[1]]
## [1] 1 1 1 1
## 
## [[2]]
## [1] 2 2 2
## 
## [[3]]
## [1] 3 3
## 
## [[4]]
## [1] 4
\end{verbatim}

\hypertarget{vectorizing-a-function}{%
\paragraph{Vectorizing a Function}\label{vectorizing-a-function}}

\begin{Shaded}
\begin{Highlighting}[]
\NormalTok{noise \textless{}{-}}\StringTok{ }\ControlFlowTok{function}\NormalTok{(n, mean, sd) \{}
    \KeywordTok{rnorm}\NormalTok{(n, mean, sd)}
\NormalTok{\}}
\end{Highlighting}
\end{Shaded}

\begin{Shaded}
\begin{Highlighting}[]
\KeywordTok{noise}\NormalTok{(}\DecValTok{5}\NormalTok{, }\DecValTok{1}\NormalTok{, }\DecValTok{2}\NormalTok{)}
\end{Highlighting}
\end{Shaded}

\begin{verbatim}
## [1] -0.5603570  0.1085129  3.0723444 -0.6130173  1.6797173
\end{verbatim}

\begin{Shaded}
\begin{Highlighting}[]
\KeywordTok{noise}\NormalTok{(}\DecValTok{1}\OperatorTok{:}\DecValTok{5}\NormalTok{, }\DecValTok{1}\OperatorTok{:}\DecValTok{5}\NormalTok{, }\DecValTok{2}\NormalTok{)}
\end{Highlighting}
\end{Shaded}

\begin{verbatim}
## [1] 4.340264 3.648094 1.564776 1.075283 5.245968
\end{verbatim}

显然以上不是我们想要的结果

\hypertarget{instant-vectorization}{%
\paragraph{Instant Vectorization}\label{instant-vectorization}}

\begin{Shaded}
\begin{Highlighting}[]
\KeywordTok{mapply}\NormalTok{(noise, }\DecValTok{1}\OperatorTok{:}\DecValTok{5}\NormalTok{, }\DecValTok{1}\OperatorTok{:}\DecValTok{5}\NormalTok{, }\DecValTok{2}\NormalTok{)}
\end{Highlighting}
\end{Shaded}

\begin{verbatim}
## [[1]]
## [1] 2.645621
## 
## [[2]]
## [1] 0.08007882 2.34264517
## 
## [[3]]
## [1]  4.0533284 -0.2985937  1.7688980
## 
## [[4]]
## [1] 8.172823 5.934735 2.952229 4.173631
## 
## [[5]]
## [1] 5.8707446 2.6804398 7.7860345 4.6545764 0.4561946
\end{verbatim}

\begin{Shaded}
\begin{Highlighting}[]
\NormalTok{?factor}
\end{Highlighting}
\end{Shaded}

\begin{verbatim}
## starting httpd help server ... done
\end{verbatim}

\hypertarget{tapply}{%
\subsubsection{tapply}\label{tapply}}

\textbf{tapply} is used to apply a function over subsets of a vector.

(Don't know why it's called **tapply*.)

\begin{Shaded}
\begin{Highlighting}[]
\KeywordTok{str}\NormalTok{(apply)}
\end{Highlighting}
\end{Shaded}

\begin{verbatim}
## function (X, MARGIN, FUN, ...)
\end{verbatim}

\begin{itemize}
\tightlist
\item
  \textbf{X} is a vector
\item
  \textbf{INDEX} is a factor or a list of factors (or else they are
  coerced to factor)
\item
  \textbf{FUN} is a fucntion to be applied
\item
  .. contains other arguments to be passed \textbf{FUN}
\item
  \textbf{simplify}, should we simplify the result?
\end{itemize}

Take group means.

\begin{Shaded}
\begin{Highlighting}[]
\NormalTok{x \textless{}{-}}\StringTok{ }\KeywordTok{c}\NormalTok{(}\KeywordTok{rnorm}\NormalTok{(}\DecValTok{10}\NormalTok{), }\KeywordTok{runif}\NormalTok{(}\DecValTok{10}\NormalTok{), }\KeywordTok{rnorm}\NormalTok{(}\DecValTok{10}\NormalTok{, }\DecValTok{1}\NormalTok{))}
\NormalTok{f \textless{}{-}}\StringTok{ }\KeywordTok{gl}\NormalTok{(}\DecValTok{3}\NormalTok{, }\DecValTok{10}\NormalTok{)}
\NormalTok{f}
\end{Highlighting}
\end{Shaded}

\begin{verbatim}
##  [1] 1 1 1 1 1 1 1 1 1 1 2 2 2 2 2 2 2 2 2 2 3 3 3 3 3 3 3 3 3 3
## Levels: 1 2 3
\end{verbatim}

\begin{Shaded}
\begin{Highlighting}[]
\KeywordTok{tapply}\NormalTok{(x, f, mean)}
\end{Highlighting}
\end{Shaded}

\begin{verbatim}
##         1         2         3 
## 0.4228764 0.3981082 1.2009519
\end{verbatim}

Take group means without simplification.

\begin{Shaded}
\begin{Highlighting}[]
\KeywordTok{tapply}\NormalTok{(x, f, mean, }\DataTypeTok{simplify =} \OtherTok{FALSE}\NormalTok{)}
\end{Highlighting}
\end{Shaded}

\begin{verbatim}
## $`1`
## [1] 0.4228764
## 
## $`2`
## [1] 0.3981082
## 
## $`3`
## [1] 1.200952
\end{verbatim}

Find group ranges.

\begin{Shaded}
\begin{Highlighting}[]
\KeywordTok{tapply}\NormalTok{(x, f, range) }\CommentTok{\# gives the min and the max of the observations within the subsets of the vector x}
\end{Highlighting}
\end{Shaded}

\begin{verbatim}
## $`1`
## [1] -2.223016  3.185123
## 
## $`2`
## [1] 0.04237351 0.95993802
## 
## $`3`
## [1] -0.5526404  2.2998786
\end{verbatim}

\hypertarget{split}{%
\subsubsection{split}\label{split}}

split takes a vector or other objects and split it into groups
determined by a factor or list of factors.

split is not a loop function but it is a very handy function that can be
used in conjuction with functions like lapply.

\begin{Shaded}
\begin{Highlighting}[]
\KeywordTok{str}\NormalTok{(split)}
\end{Highlighting}
\end{Shaded}

\begin{verbatim}
## function (x, f, drop = FALSE, ...)
\end{verbatim}

\begin{itemize}
\tightlist
\item
  \textbf{x} is a vector (or list) or data frame
\item
  \textbf{f} is a factor (or coerced to one) or a list of factor
\item
  \textbf{drop} indicates whether empty factors levels should be dropped
\end{itemize}

\begin{Shaded}
\begin{Highlighting}[]
\NormalTok{x \textless{}{-}}\StringTok{ }\KeywordTok{c}\NormalTok{(}\KeywordTok{rnorm}\NormalTok{(}\DecValTok{10}\NormalTok{), }\KeywordTok{runif}\NormalTok{(}\DecValTok{10}\NormalTok{), }\KeywordTok{rnorm}\NormalTok{(}\DecValTok{10}\NormalTok{, }\DecValTok{1}\NormalTok{))}
\NormalTok{f \textless{}{-}}\StringTok{ }\KeywordTok{gl}\NormalTok{(}\DecValTok{3}\NormalTok{, }\DecValTok{10}\NormalTok{)}
\KeywordTok{split}\NormalTok{(x, f)}
\end{Highlighting}
\end{Shaded}

\begin{verbatim}
## $`1`
##  [1] -1.70561930  0.95897395  1.06864208  0.60754809 -0.93774493  0.70752373
##  [7]  0.33097228  0.83777992  0.19176928 -0.01519066
## 
## $`2`
##  [1] 0.5021744 0.2208294 0.9559347 0.9026626 0.4919914 0.2159218 0.8203364
##  [8] 0.3918720 0.8471233 0.1260708
## 
## $`3`
##  [1]  0.1766738  1.7399116 -0.3286700  1.1139291  1.3632336  2.6966651
##  [7]  1.7888473  0.9690902  0.4650057  1.8579001
\end{verbatim}

\begin{Shaded}
\begin{Highlighting}[]
\KeywordTok{lapply}\NormalTok{(}\KeywordTok{split}\NormalTok{(x, f), mean)}
\end{Highlighting}
\end{Shaded}

\begin{verbatim}
## $`1`
## [1] 0.2044654
## 
## $`2`
## [1] 0.5474917
## 
## $`3`
## [1] 1.184259
\end{verbatim}

\begin{Shaded}
\begin{Highlighting}[]
\KeywordTok{tapply}\NormalTok{(x, f, mean)}
\end{Highlighting}
\end{Shaded}

\begin{verbatim}
##         1         2         3 
## 0.2044654 0.5474917 1.1842587
\end{verbatim}

\hypertarget{splitting-a-data-frame}{%
\paragraph{Splitting a Data Frame}\label{splitting-a-data-frame}}

\begin{Shaded}
\begin{Highlighting}[]
\KeywordTok{library}\NormalTok{(datasets)}
\KeywordTok{head}\NormalTok{(airquality)}
\end{Highlighting}
\end{Shaded}

\begin{verbatim}
##   Ozone Solar.R Wind Temp Month Day
## 1    41     190  7.4   67     5   1
## 2    36     118  8.0   72     5   2
## 3    12     149 12.6   74     5   3
## 4    18     313 11.5   62     5   4
## 5    NA      NA 14.3   56     5   5
## 6    28      NA 14.9   66     5   6
\end{verbatim}

\begin{Shaded}
\begin{Highlighting}[]
\NormalTok{s \textless{}{-}}\StringTok{ }\KeywordTok{split}\NormalTok{(airquality, airquality}\OperatorTok{$}\NormalTok{Month)}
\KeywordTok{lapply}\NormalTok{(s, }\ControlFlowTok{function}\NormalTok{(x) }\KeywordTok{colMeans}\NormalTok{(x[, }\KeywordTok{c}\NormalTok{(}\StringTok{"Ozone"}\NormalTok{, }\StringTok{"Solar.R"}\NormalTok{, }\StringTok{"Wind"}\NormalTok{)]))}
\end{Highlighting}
\end{Shaded}

\begin{verbatim}
## $`5`
##    Ozone  Solar.R     Wind 
##       NA       NA 11.62258 
## 
## $`6`
##     Ozone   Solar.R      Wind 
##        NA 190.16667  10.26667 
## 
## $`7`
##      Ozone    Solar.R       Wind 
##         NA 216.483871   8.941935 
## 
## $`8`
##    Ozone  Solar.R     Wind 
##       NA       NA 8.793548 
## 
## $`9`
##    Ozone  Solar.R     Wind 
##       NA 167.4333  10.1800
\end{verbatim}

\begin{Shaded}
\begin{Highlighting}[]
\KeywordTok{sapply}\NormalTok{(s, }\ControlFlowTok{function}\NormalTok{(x) }\KeywordTok{colMeans}\NormalTok{(x[, }\KeywordTok{c}\NormalTok{(}\StringTok{"Ozone"}\NormalTok{, }\StringTok{"Solar.R"}\NormalTok{, }\StringTok{"Wind"}\NormalTok{)]))}
\end{Highlighting}
\end{Shaded}

\begin{verbatim}
##                5         6          7        8        9
## Ozone         NA        NA         NA       NA       NA
## Solar.R       NA 190.16667 216.483871       NA 167.4333
## Wind    11.62258  10.26667   8.941935 8.793548  10.1800
\end{verbatim}

\begin{Shaded}
\begin{Highlighting}[]
\KeywordTok{sapply}\NormalTok{(s, }\ControlFlowTok{function}\NormalTok{(x) }\KeywordTok{colMeans}\NormalTok{(x[, }\KeywordTok{c}\NormalTok{(}\StringTok{"Ozone"}\NormalTok{, }\StringTok{"Solar.R"}\NormalTok{, }\StringTok{"Wind"}\NormalTok{)], }\DataTypeTok{na.rm =} \OtherTok{TRUE}\NormalTok{))}
\end{Highlighting}
\end{Shaded}

\begin{verbatim}
##                 5         6          7          8         9
## Ozone    23.61538  29.44444  59.115385  59.961538  31.44828
## Solar.R 181.29630 190.16667 216.483871 171.857143 167.43333
## Wind     11.62258  10.26667   8.941935   8.793548  10.18000
\end{verbatim}

\hypertarget{splitting-on-more-than-one-level}{%
\paragraph{Splitting on More than One
Level}\label{splitting-on-more-than-one-level}}

\begin{Shaded}
\begin{Highlighting}[]
\NormalTok{x \textless{}{-}}\StringTok{ }\KeywordTok{rnorm}\NormalTok{(}\DecValTok{10}\NormalTok{)}
\NormalTok{f1 \textless{}{-}}\StringTok{ }\KeywordTok{gl}\NormalTok{(}\DecValTok{2}\NormalTok{, }\DecValTok{5}\NormalTok{)}
\NormalTok{f2 \textless{}{-}}\StringTok{ }\KeywordTok{gl}\NormalTok{(}\DecValTok{5}\NormalTok{, }\DecValTok{2}\NormalTok{)}
\NormalTok{f1}
\end{Highlighting}
\end{Shaded}

\begin{verbatim}
##  [1] 1 1 1 1 1 2 2 2 2 2
## Levels: 1 2
\end{verbatim}

\begin{Shaded}
\begin{Highlighting}[]
\NormalTok{f2}
\end{Highlighting}
\end{Shaded}

\begin{verbatim}
##  [1] 1 1 2 2 3 3 4 4 5 5
## Levels: 1 2 3 4 5
\end{verbatim}

\begin{Shaded}
\begin{Highlighting}[]
\KeywordTok{interaction}\NormalTok{(f1, f2)}
\end{Highlighting}
\end{Shaded}

\begin{verbatim}
##  [1] 1.1 1.1 1.2 1.2 1.3 2.3 2.4 2.4 2.5 2.5
## Levels: 1.1 2.1 1.2 2.2 1.3 2.3 1.4 2.4 1.5 2.5
\end{verbatim}

\begin{Shaded}
\begin{Highlighting}[]
\KeywordTok{interaction}\NormalTok{(f1, f2) }\CommentTok{\# 十个factor一一对应左右对应组合起来的结果}
\end{Highlighting}
\end{Shaded}

\begin{verbatim}
##  [1] 1.1 1.1 1.2 1.2 1.3 2.3 2.4 2.4 2.5 2.5
## Levels: 1.1 2.1 1.2 2.2 1.3 2.3 1.4 2.4 1.5 2.5
\end{verbatim}

Interactions can create empty levels.

\begin{Shaded}
\begin{Highlighting}[]
\KeywordTok{str}\NormalTok{(}\KeywordTok{split}\NormalTok{(x, }\KeywordTok{list}\NormalTok{(f1, f2)))}
\end{Highlighting}
\end{Shaded}

\begin{verbatim}
## List of 10
##  $ 1.1: num [1:2] -1.25 -1.64
##  $ 2.1: num(0) 
##  $ 1.2: num [1:2] -0.102 -2.855
##  $ 2.2: num(0) 
##  $ 1.3: num 0.081
##  $ 2.3: num -1.69
##  $ 1.4: num(0) 
##  $ 2.4: num [1:2] 1.93 -0.0708
##  $ 1.5: num(0) 
##  $ 2.5: num [1:2] -1.552 -0.989
\end{verbatim}

Empty levels can be dropped.

\begin{Shaded}
\begin{Highlighting}[]
\KeywordTok{str}\NormalTok{(}\KeywordTok{split}\NormalTok{(x, }\KeywordTok{list}\NormalTok{(f1, f2), }\DataTypeTok{drop =} \OtherTok{TRUE}\NormalTok{))}
\end{Highlighting}
\end{Shaded}

\begin{verbatim}
## List of 6
##  $ 1.1: num [1:2] -1.25 -1.64
##  $ 1.2: num [1:2] -0.102 -2.855
##  $ 1.3: num 0.081
##  $ 2.3: num -1.69
##  $ 2.4: num [1:2] 1.93 -0.0708
##  $ 2.5: num [1:2] -1.552 -0.989
\end{verbatim}

\begin{Shaded}
\begin{Highlighting}[]
\KeywordTok{list}\NormalTok{(f1, f2)}
\end{Highlighting}
\end{Shaded}

\begin{verbatim}
## [[1]]
##  [1] 1 1 1 1 1 2 2 2 2 2
## Levels: 1 2
## 
## [[2]]
##  [1] 1 1 2 2 3 3 4 4 5 5
## Levels: 1 2 3 4 5
\end{verbatim}

\hypertarget{debugging-tools-in-r}{%
\subsubsection{Debugging Tools in R}\label{debugging-tools-in-r}}

The primary tools for debugging functions in R are

\begin{itemize}
\tightlist
\item
  \textbf{traceback}: prints out the function call stack after an error
  occurs; does nothing if there's no error
\item
  \textbf{debug}: flags a function for ``debug'' mode which allows you
  to step through execution of a function one line at a time
\item
  \textbf{browser}: suspends the execution of a function wherever it is
  called and puts the function in debug mode
\item
  \textbf{trace}: allows you to insert debugging code into a function a
  specific places
\item
  \textbf{recover}: allows you to modify the error behavior so that you
  can browse the function call stack
\end{itemize}

\hypertarget{trackback}{%
\paragraph{trackback}\label{trackback}}

当发生错误时,traceback打印出错函数的函数调用栈

\begin{Shaded}
\begin{Highlighting}[]
\CommentTok{\# mean(z)}
\end{Highlighting}
\end{Shaded}

\begin{Shaded}
\begin{Highlighting}[]
\KeywordTok{traceback}\NormalTok{()}
\end{Highlighting}
\end{Shaded}

\begin{verbatim}
## No traceback available
\end{verbatim}

\begin{Shaded}
\begin{Highlighting}[]
\CommentTok{\# lm(y {-} x)}
\end{Highlighting}
\end{Shaded}

\begin{Shaded}
\begin{Highlighting}[]
\KeywordTok{traceback}\NormalTok{()}
\end{Highlighting}
\end{Shaded}

\begin{verbatim}
## No traceback available
\end{verbatim}

\hypertarget{debug}{%
\paragraph{debug}\label{debug}}

\begin{Shaded}
\begin{Highlighting}[]
\KeywordTok{debug}\NormalTok{(lm)}
\end{Highlighting}
\end{Shaded}

\begin{Shaded}
\begin{Highlighting}[]
\CommentTok{\# lm(z {-} y)}
\end{Highlighting}
\end{Shaded}

此时调用lm函数,会先打印出整个函数的代码,然后进入browse模式,可以一行一行的执行代码,定位具体出错的位置。

\hypertarget{recover}{%
\paragraph{recover}\label{recover}}

可以通过options()把recover函数设为错误处理器

\begin{Shaded}
\begin{Highlighting}[]
\NormalTok{options}
\end{Highlighting}
\end{Shaded}

\begin{verbatim}
## function (...) 
## .Internal(options(...))
## <bytecode: 0x0000000012f17fa0>
## <environment: namespace:base>
\end{verbatim}

\end{document}
